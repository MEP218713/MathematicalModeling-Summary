% !Mode:: "TeX:UTF-8"
% !TEX program  = xelatex

\documentclass{cumcmthesis}
%\documentclass[withoutpreface,bwprint]{cumcmthesis} %去掉封面与编号页,电子版提交的时候使用。

\usepackage[framemethod=TikZ]{mdframed}
\usepackage{url}   % 网页链接
\usepackage{subcaption} % 子标题
\title{基于TOPSIS法和模糊综合评判的中小微企业信贷评估与策略}
\tihao{C}
\baominghao{202022006088}
\schoolname{西南大学}
\membera{ }
\memberb{ }
\memberc{ }
\supervisor{ }
\yearinput{2020}
\monthinput{09}
\dayinput{14}


\begin{document}

\maketitle
\begin{abstract}
中小微型企业不但是技术企业的摇篮,还为新型人才创造大量就业机遇,有力促进了国民经济发展,为社会稳定繁荣助力,此外还对改善民生起到了重要作用。然而它们的发展和经营受到层层阻碍。在实际中,由于中小微企业规模相对较小,缺少抵押资产,因此银行通常向实力强、供求关系稳定的企业提供贷款,并对信誉高、信贷风险小的企业给予利率优惠,因此向银行研究对中小微企业的信贷策略具有深远意义。本文针对特定的企业数据和银行的基本贷款政策对该问题进行了探究。

针对问题一,本文首先对企业的信誉评价用具体的数据表示,依据123家企业的进项销项发票信息分类评估,建立信誉评级与交易票据的相关关系,为问题二做准备;考虑的关键因素是量化的标准化和归一化,因此选用TOPSIS算法分别对不同信誉评级的企业计算有效发票比,月平均利润和收益比等数值指标,建立信誉风险指数模型。对信誉评级和企业综合实力评估建立联系和对应后,先筛去D级企业,第二次删选只需要在信誉评级为A,B和C的企业中删去总利润为负数的企业,再运用TOPSIS算法建立贷款配额模型,以及运用数学最优化理论建立权重附给利率制定模型对允许贷款的企业设计借贷策略。

针对问题二,则是通过问题一得出的信誉评级和企业综合实力评估的对应关系对企业进行信誉预测评级,估计附件中302家企业的信誉评级,首先筛去D级企业,同时运用问题一的TOPSIS算法三个模型,在银行一亿元贷款总额下做出类似的贷款配额和利率指配策略。结果显示通过三次信贷额的分配,企业得到最终的年信贷额分配,信誉等级为A、B、C的贷款年利率分别为4.65$\%$,5.85$\%$,5.85$\%$。

针对问题三,目标是评估突发因素对各企业的影响来给出该银行在年度信贷总额为1亿元时的信贷调整策略。首先对302家企业进行粗略的分类,分为7类,不同企业类型应对突发因素的平均能力由查文献可得。对分类后的企业提取因素集,确定各因素的权重,建立模糊综合评价模型求得以这302家企业为样本的不同类型企业的综合应对能力等级。银行根据所得的不同类型企业的综合应对能力等级对不同企业类型作再次评估,设定合理的指标,对综合应对能力等级不同的企业作适当的贷款额和利率增减。

\keywords{TOPSIS算法\quad 银行配额决策\quad  中小微企业\quad  模糊综合评价}
\end{abstract}

%\tableofcontents
%\newpage
\section{问题重述}
中小微型企业在世界各地区和各国经济当中占据着不容小觑的地位,从2010年起,我国的中小微型企业如雨后春笋大量涌现,中小微型企业不但是技术企业的摇篮,还为新型人才创造大量就业机遇,有力促进了国民经济发展,为社会稳定繁荣助力,此外还对改善民生起到了重要作用。然而它们的发展和经营受到层层阻碍。在实际中,由于中小微企业规模相对较小,缺少抵押资产,因此银行通常向实力强、供求关系稳定的企业提供贷款,并对信誉高、信贷风险小的企业给予利率优惠。信贷决策问题是银行重视的问题之一,银行需要进行一定的判断才能决定是否给予中小型微企业贷款,因此通过数学建模研究给中小型微企业的信贷策略非常具有意义与价值。

本文我们需要解决的问题有:

一是根据有过信贷记录的企业的数据选择一个综合评价方法进行数值量化分析,进行综合借贷风险评估,通过分析对各个企业的借贷风险指数作出是否给予借贷资格的决策,进而设计模型给出该银行在年度信贷总额固定时对微小企业的信贷策略;

二是根据问题一中的方法实现对无信贷记录企业进行借贷信誉评估和借贷风险评估,并给出相应的信贷策略;

三是在企业的实际经营中往往有一些突发因素会影响到企业的发展,选择部分重要因素作为影响企业发展的突发因素和题目二中量化得到的企业信贷风险综合考虑给出银行信贷调整策略。
\section{问题分析}
\subsection{对问题一的分析}
本小问需要我们从对附件一中123家企业的信贷风险进行评估。考虑到问题二并没有企业的信贷记录相关信息,故我们首先依据附件一的信誉评级和进项、销项发票信息,对企业的信誉评价和企业交易票据信息对应联系,建立建立信誉风险指数模型,有依据地用具体的数值表示信誉评级。分别对不同信誉评级的企业利用附件一中地发票数据计算有效发票比,月平均交易利润,收益比等数值指标,考虑的关键因素是量化的标准化和归一化。对不同信誉评级的企业,确定各个指标的权重,运用TOPSIS算法得到一个信贷风险得分,再将信誉风险指数和企业信誉评级进行设计对应,计算出不同信誉评级的企业的平均指标值作为评估无信贷记录企业的依据。

对信誉评级和企业综合实力评估建立联系和对应后,先筛去D级企业,第二次删选只需要在信誉评级为A,B和C的企业中删去总利润为负数的企业,再运用TOPSIS算法建立贷款配额模型,以及运用数学最优化理论建立权重附给利率指配模型对允许贷款的企业设计借贷策略。

截去不予贷款的企业后,再运用TOPSIS算法建立贷款配额模型,对余下允许贷款的企业进行贷款额度分配。考虑的关键因素是各个企业的收益比,平均年度需贷额(用平均年度购买成本额减去$\frac{1}{10}$的平均年度纯利润额)以及利用前面所得的信誉风险指数这三方面,先对收益比和信誉风险指数赋主观权重,设计出成对比较判断矩阵,计算出贷款指数,按得分高低一次分配贷款额(银行贷款总额一定),对于被分配贷款额大于平均年度需贷额的企业,截取它们出多余的贷款额,将剩下的贷款额上述用类似的方法进行二次分配,三次分配...直至贷款额分配完毕。

在求解发放贷款的利率时,我们建立了一个银行最优收益模型,以银行第二年的收益为目标,建立目标函数。在这里,我们考虑了银行贷款年利率与不同信誉等级的客户流失率的关系,在已知客户流失率的前提下,通过穷举法来考虑可能的不同信誉等级的客户流失率的组合,获得不同信誉等级客户流失率组合下的银行收益,找到最优的银行收益,从而得到不同信誉等级客户流失率的最优组合,最后根据这一组合查表得出企业不同信誉等级的贷款年利率。

\subsection{对问题二的分析}
在问题一的基础上,针对问题二,鉴于该题中302家中小微企业均为无信贷记录的企业,所以需要对其信誉等级进行评估。基本方法是根据302家企业的实力来评估其信誉等级。第一问中的123家中小微企业有信贷记录和信誉等级,所以利用TOPSIS算法(即信誉评分模型)和第一问数据计算出信誉等级为A、B、C以及D时的评级范围标准,然后根据这一标准对302家企业的信誉进行评级。筛选掉信誉等级为D的企业,之后在利用信誉等级为A、B和C的企业的信誉得分、企业收益比、平均年度成本额以及进项和销项的平均有效发票比这四个因素得到一亿元贷款的分配额得分,在这里建立的是基于TOPSIS算法的贷款配额得分模型。

根据配额得分计算每一个企业的贷款分配比例,用该比例乘以总的贷款额(一亿元)即为企业第一次预分配的贷款额,由于不同企业的平均年度需贷款额不同,所以需要贷款的数额也会有所不同。在这里,我们计算第一次预分配的贷款额与企业所需要的贷款额的差值,若该值大于零,则认为预分配的贷款额超过了企业所需要的贷款额,并且这一部分的数额将用于之后的再次分配,因此将企业所需要的贷款额作为企业最终获得的贷款额;若该值小于零,则认为在第一次的分配并未达到企业所需要的贷款额。完成第一次的分配之后,调整剩余企业的贷款分配比例,计算二次分配贷款额,由此筛选出满足企业贷款需求的企业,之后又进行下一次迭代,直到贷款总额被完全分配给企业或者所有的企业均获得满足企业需求的贷款额结束。

至于利率的制定,与问题一的分析一致,只需要将银行的贷款总额一亿元代入求出具体指定的利率即可。

\subsection{对问题三的分析}
问题的解决目标是评估突发因素对各企业的影响来给出该银行在年度信贷总额为1亿元时的信贷调整策略。考虑到不同企业类型应对突发风险能力有较大差距,首先对302家企业进行粗略的分类,分为7类,不同企业类型应对突发因素的平均能力由查文献可得。对分类后的企业计算出月平均发票数、月平均利润,加上问题二所求得的信贷风险指数以及查文献所得的平均应对能力,将四者作为因素集,确定各因素的权重,建立模糊综合评价模型求得以这302家企业为样本的不同类型企业的综合应对能力等级。

银行根据所得的不同类型企业的综合应对能力等级对不同企业类型作再次评估,设定合理的指标,对综合应对能力差的企业作适当的贷款额缩减和利率提升,为保持银行总贷款额恒定,将所有缩减后剩下的贷款额综合应对能力好的企业作适当的贷款增额和利率降低。



\section{模型的假设与符号说明}

\subsection{模型假设}
\begin{enumerate}
	\item 题目所给数据具有可靠性和有效性;
	\item 计算平均年贷款额中赋给的利润投资比例贴近现实;
	\item 算法中主观赋给的指标权重贴近现实;
	\item 银行固定的所给年度信贷总额不受任何社会经济风险影响而改变;
	\item 题目所给的企业在给予决策的短期内较先前不会有较大变故。
\end{enumerate}

\let\cleardoublepage\clearpage

\subsection{符号说明}
\begin{table}[H]   %[H]
 \centering
	\begin{tabular}{cc}
		\toprule[1.5pt]
		符号 & 含义\\
		\midrule[1pt]
		$m$ & 企业数\\
		$n$ & 指标数\\
		$E$ & 成对比较判断矩阵\\
		$A$ & 决策矩阵\\
		$B$ & 规范化决策矩阵\\
		$\lambda$ & 特征值\\
		\bottomrule[1.5pt]
\end{tabular}
\end{table}	

\begin{table}[H]   %[H]
	\centering
	\begin{tabular}{cc}
		\toprule[1.5pt]
		符号 & 含义\\
		\midrule[1pt]		
		$w$ & 归一化特征向量\\
		$C^*$ & 正理想解\\
		$C^0$ & 负理想解\\
		$s_i^*$ & 各企业到正理想解与负理想解的距离\\
		$s_i^0$ & 备选企业$d_i$到负理想解的距离\\
		$f_i^*$ & 各企业评价指数\\
		$\Phi$ & 银行总贷额\\
		$r_i$ & 第i次预分配额 \\
		$x_n$ & 信誉A企业的流失率 \\
		$y_n$ & 信誉为B企业的流失率 \\
		$z_n$ & 信誉为C企业的流失率 \\
		$k$ & 指标占比总和 \\
		$X$ & 信誉A企业的贷款利率\\
		$Y$ & 信誉B企业的贷款利率\\
		$Z$ & 信誉C企业的贷款利率\\
		$a_n$ & 信誉A企业的总贷款分配额\\
		$c_n$ &信誉B企业的总贷款分配额\\	
		$b_n$ &信誉C企业的总贷款分配额\\
		$W_j^i$ & 第j家企业在第i次分配中该企业的贷款分配占比\\
		$R_j^i$ & 第j家企业在第i次分配时的待分配信贷额\\
		$U$ & 因素集 \\
		$Q$ & 模糊向量 \\
		$R$ & 模糊综合判断矩阵 \\
		$B$ & 综合评判结果 \\
		
		\bottomrule[1.5pt]
	\end{tabular}
\end{table}

\section{模型准备}
\subsection{TOPSIS算法模型原理}
理想解法,亦称为TOPSIS法,是一种高效的多指标评价方法,能充分利用原始数据的信息,其结果能精确地反映各评价方案之间的差距。用TOPSIS算法求解多指标决策问题的理念很简单,只要在指标空间中定义适当的距离测度就能够计算备选方案与理想解的距离。TOPSIS法所用的距离是欧式距离。至于既用正理想解又用负理想解是因为在仅仅使用正理想解时有时候会产生2个备选方案与正理想解的距离相同的情况,为了区分这两个方案的优劣,我们引入负理想解并计算2个方案与负理想解的距离,与正理想解的距离相同的方案离负理想解远者为最优。

\section{模型的建立与求解}
\subsection{问题一的模型的建立与求解}

\subsubsection{数据处理与筛选}
分别对不同信誉评级的企业利用附件一中地发票数据计算有效发票比,月平均交易利润,收益比等数值指标,考虑的关键因素是量化的标准化和归一化。附录中\cref{tab:001}为根据附件1表计算的信誉评级为A的企业的相关实力指标数据(信誉评级为B,C和D的企业的相关实力指标数据也在附录中给出).

其中, 每个企业的平均有效发票比的计算公式为
\begin{equation}
\mbox{平均有效发票比}=\frac{1}{2}(\frac{\mbox{进项有效发票数}}{\mbox{进项总发票数}}+\frac{\mbox{销项有效发票数}}{\mbox{销项总发票数}})
\end{equation}

月平均利润的计算公式为
\begin{align}
\mbox{月平均利润} &= \frac{\mbox{利润}}{\mbox{所含发票的月份数}}\\
                 &= \frac{\mbox{销项价税合计}+\mbox{进项价税合计}-\mbox{销项税额}+\mbox{进项税额}}{\mbox{所含发票的月份数}}
\end{align}

收益比的计算公式为
\begin{align}
\mbox{收益比} &= \frac{\mbox{利润}}{\mbox{进项价税合计}} \\
             &=\frac{\mbox{销项价税合计}+\mbox{进项价税合计}-\mbox{销项税额}+\mbox{进项税额}}{\mbox{进项价税合计}}
\end{align}

\subsubsection{信誉风险指数模型}
我们利用附件一中的发票数据计算有效发票比,月平均交易利润,收益比等数值指标后,基于TOPSIS算法来建立信誉风险指数模型。现在我们先对信誉评级为A的企业估计信誉风险指数。

对月平均交易利润为负数的企业作淘汰处理后,余下企业是可进行贷款的信誉评级为A的企业:(E2,E6,E7,E8,E9,E13,E15,E16,E17,E18,E22,E24,E31,E42,E48,E54,E59,E64)

决策矩阵$A=(A_{ij})_{m\times n}$在附录中,用向量规划化的方法求得规范决策矩阵,得到规范化决策矩阵
\begin{align}
B &=(b_{ij})_{m\times n}\\
  &= \frac{a_{ij}}{\sqrt{\sum_{i=1}^m a_{ij}^2}},i=1,...,m;j=1,...,n
\end{align}

我们计算得:

\[
\mathbf{B} = \left(
\begin{array}{ccc}
-0.5773504   & 1.154700538 & -0.577350138 \\
-0.577349914 & 1.154700538 & -0.577350624 \\
-0.577350717 & 1.154700538 & -0.577349822 \\
-0.577350307 & 1.154700538 & -0.577350231 \\
-0.57735141  & 1.154700538 & -0.577349128 \\
-0.577350336 & 1.154700538 & -0.577350202 \\
-0.577353794 & 1.154700538 & -0.577346745 \\
-0.577427095 & 1.154700535 & -0.57727344  \\
-0.577349335 & 1.154700538 & -0.577351204 \\
-0.577350006 & 1.154700538 & -0.577350533 \\
-0.577350434 & 1.154700538 & -0.577350105 \\
-0.577350175 & 1.154700538 & -0.577350364 \\
-0.577355878 & 1.154700538 & -0.57734466  \\
-0.580073897 & 1.154696249 & -0.574622352 \\
-0.577351713 & 1.154700538 & -0.577348826 \\
-0.577353128 & 1.154700538 & -0.577347411 \\
-0.577348108 & 1.154700538 & -0.57735243  \\
-0.57753002  & 1.15470052  & -0.5771705   \\
-0.577342915 & 1.154700538 & -0.577357624 \\
-0.577355364 & 1.154700538 & -0.577345175 \\
-0.577380855 & 1.154700538 & -0.577319682 \\
-0.577342266 & 1.154700538 & -0.577358272\\
\end{array} \right)
\]

企业的综合信誉风险指数的确定采用指标成绩加权求和,权值的确定以采用层次分析法的思想,在这种方法中,需要建立成对比较判断矩阵,设成对比较判断矩阵(建模者主观给出的矩阵)为

\[
\mathbf{E} = \left(
\begin{array}{ccc}
1 &  \frac{1}{3} & \frac{1}{4}\\
3 &  1 & 7\\
4 &  \frac{1}{7} & 1\\
\end{array} \right)
\]

要求各个指标的权重向量为$\textbf{w}=[w_1,w_2,...,w_n]^T$,则只需要求出成对比较判断矩阵$\textbf{E}$的最大特征值$\lambda=3.580$,那么其对应的归一化特征向量为
\begin{equation}
\textbf{w}=[-0.0749 + 0.1298i,	0.9468+ 0.0i,-0.1423-0.2466i ]^T
\end{equation}
即得到3个指标对应的权重。

然后确定正理想解$C^*$和负理想解$C^0$。设正理想解$C^*$的第j个指标值为$c_j^*$,负理想解$C^0$第j个指标值为$c_j^0$,则
\begin{equation}
c_j^*=max_i c_{ij},j=1,...,n
\end{equation}
\begin{equation}
c_j^0=min_i c_{ij},j=1,...,n
\end{equation}

备选企业$d_i$到正理想解的距离为:
\begin{equation}
s_i^*=\sqrt{\sum_{j=1}^n(c_ij-c_j^*)^2},i=1,2,...,m;
\end{equation}

备选企业$d_i$到负理想解的距离为:
\begin{equation}
s_i^0=\sqrt{\sum_{j=1}^n(c_ij-c_j^0)^2},i=1,2,...,m;
\end{equation}

最后计算各企业的排队指标值(即综合评价指数),即
\begin{equation}
f_j^*=\frac{s_i^0}{s_i^0+s_i^*},i=1,...,m
\end{equation}
最后允许贷款的信誉为A的企业的距离值以及综合指标值结果如下表:

\begin{table}[H]   %[H]
	\caption{附件一中允许贷款的信誉为A的企业的距离值以及综合指标值}\label{tab:010} \centering
	\begin{tabular}{cccc}
		\toprule[1.5pt]
		企业 & $s_i^*$    & $s_i^0$        & $f_i^*$         \\
		\midrule[1pt]
E2 & 0.00040825 & 0.000776825 & 0.65550723 \\ 
E6 & 0.00040825 & 0.000776963 & 0.655547425 \\ 
E7 & 0.00040825 & 0.000776735 & 0.655481037 \\ 
E8 & 0.00040825 & 0.000776851 & 0.655514917 \\ 
E9 & 0.00040825 & 0.000776538 & 0.655423679 \\ 
E13 & 0.00040825 & 0.000776843 & 0.655512518 \\ 
E15 & 0.00040825 & 0.000775859 & 0.655226334 \\ 
E16 & 0.00040825 & 0.000755089 & 0.649070824 \\ 
E17 & 0.00040825 & 0.000777128 & 0.655595332 \\ 
E18 & 0.00040825 & 0.000776937 & 0.655539844 \\ 
E22 & 0.00040825 & 0.000776815 & 0.655504445 \\ 
E24 & 0.00040825 & 0.000776889 & 0.655525878 \\ 
E31 & 0.00040825 & 0.000775267 & 0.655053662 \\
E42 & 0.00040825 & 0.000409462 & 0.500741501 \\
E48 & 0.00040825 & 0.000776451 & 0.655398632 \\ 
E54 & 0.00040825 & 0.000776049 & 0.655281487 \\ 
E59 & 0.00040825 & 0.000777477 & 0.65569671 \\ 
E64 & 0.00040825 & 0.000726213 & 0.640138359 \\
E81 & 0.00040825 & 0.000778956 & 0.656125496 \\
E84 & 0.00040825 & 0.000775413 & 0.655096269 \\ 
E88 & 0.00040825 & 0.000768173 & 0.652973582 \\ 
E91 & 0.00040825 & 0.00077914 & 0.656178972 \\ 
		\bottomrule[1.5pt]
\end{tabular}
\end{table}
允许贷款的信誉为B,C和D企业的距离值以及综合指标值结果分别放在附录。

计算出不同信誉评级的企业的平均指标值作为评估无信贷记录企业的依据:
\begin{table}[H]   %[H]
	\caption{不同信誉评级的企业的平均指标值}\label{tab:16} \centering
	\begin{tabular}{cc}
		\toprule[1.5pt]
		信誉评级的企业 & 平均指标值 \\
		\midrule[1pt]
		A & 0.6473\\
		B & 0.6459\\
		C & 0.6446\\
		D & 0.3510\\
		\bottomrule[1.5pt]
	\end{tabular}
\end{table}

\subsubsection{贷款配额模型}
由以上的模型求解,已筛去信誉评级为D的企业和平均利润为负的企业不予借贷:

\begin{table}[H]   %[H]
	\caption{附件一中不予借贷的企业}\label{tab:17} \centering
	\begin{tabular}{cccccc}
		\toprule[1.5pt]
		二次筛去企业 & 二次筛去企业 &信誉D企业&信誉D企业&信誉D企业&信誉D企业\\
		\midrule[1pt]
E1 & E21 &E36 & E112 &E102 & E118 \\ 
E19 & E33 &E52 & E113& E103 & E119\\ 
E26 & E66 &E82 & E114 &E107 & E120 \\ 
E27 & E96 &E99 & E115 & E108 & E121\\ 
E89 & E83 &E100 & E116 &E109 & E122\\ 
E20 & E101 & E117 & E111 & E123&   \\ 
		\bottomrule[1.5pt]
	\end{tabular}
\end{table}

对余下信誉评级为A,B及C的盈利企业予以贷款资格后,建立贷款配额模型进行贷款配额。我们仍然运用TOPSIS算法,用4个指标[收益比,信誉风险指数(即上一所得的综合指标值$f_i^*$),平均年度成本额,有效发票比]和另一成对比较判断矩阵计算出新的另一综合指标值$f_j^*$作为可贷款企业的评估指数。

为了得到不同信誉水平和风险水平的企业得到相应合理的贷款配额,计算每一个企业的综合指标占比

\begin{equation}
\frac{f_j^*}{\sum_{j=1}^n f_j^*},j=1,..,m
\label{eq:zhibiaobi}
\end{equation}


进而计算出每个企业的应有的贷款配额比例,该比例与企业的综合指标占比相同。设$\Phi$为银行固定贷款总额,那么每个企业第一次预分配额$r_1$为
\begin{equation}
r_1=\Phi\times\frac{f_j^*}{\sum_{j=1}^n f_j^*},j=1,..,m
\end{equation}
若企业得到的第一次预分配额大于本身的平均年度成本额或者大于银行最大贷款额度100万,那么将多余的款额截取。所有第一次预分配额多获得的企业的截去款额留作第二次预分配,第一次被截取过的企业不再容许接受第二次预分配。同理,若余下第一次没有被截取的企业的第二次预分配额大于本身的平均年度成本额或者大于银行最大贷款额度100万,那么将多余的款额截取,所有第二次预分配额多获得的企业的截去款额留作第三次预分配,以此类推,直到银行固定贷款总额$\Phi$被分配完毕,或者所有被分配的企业的需贷款额已达饱和。

\subsubsection{利率制定模型}
由贷款配额模型,我们已经得到可贷款企业的综合评估指数和每一个企业的综合指标占比,进而计算三个信誉评级企业的综合指标占比总和,将其作为三种信誉评级企业利率制定的比例关系。
\begin{table}[H]   %[H]
	\caption{附件一中不同信誉评级企业的指标占比总和}\label{tab:21} \centering
	\begin{tabular}{cc}
		\toprule[1.5pt]
		信誉评级 & 指标占比总和 k\\
		\midrule[1pt]
A & 0.18405966 \\ 
B & 0.231876586 \\ 
C & 0.584063753 \\ 
		\bottomrule[1.5pt]
\end{tabular}
\end{table}
一个基本的利率制定模型框架如下(银行总贷款定额为$\Phi$):
\begin{table}[H]   %[H]
\label{tab:22} \centering
	\begin{tabular}{cccc}
		\textbf{信誉评级} & \textbf{总贷款分配额} & \textbf{客户流失率} & \textbf{次年总贷款分配额}\\
		\midrule[1pt]
		A & $a_n$ & $x_n$ & $a_{n+1}=a_n(1-x_n)$ \\ 
		B & $b_n$ & $y_n$ & $b_{n+1}=b_n(1-y_n)$ \\ 
		C & $c_n$ & $z_n$ & $c_{n+1}=c_n(1-z_n)$ \\ 
	\end{tabular}
\end{table}

从银行的角度分析,银行是希望在分别对信誉评级为A,B和C制定了利率$X$,$Y$和$Z$后,分别存在客户流失率$x_n$,$y_n$和$z_n$,导致次年的各类企业总贷款额分别下降为
$a_{n+1}=a_n(1-x_n)$,$b_{n+1}=b_n(1-y_n)$和$c_{n+1}=c_n(1-z_n)$。

先前假设
\begin{equation}
a_n+b_n+c_n\geq\Phi
\end{equation}
银行为了保证自身的收益,自然希望在发生客户部份流失之后,仍然有
\begin{equation}
a_{n+1}+b_{n+1}+c_{n+1}\geq\Phi
\end{equation}
建立银行最优收益的目标函数为:
\begin{equation}
F(X,Y,Z)=a_n(1-x_n)X+b_n(1-y_n)Y+c_n(1-z_n)Z
\end{equation}
我们只需要通过穷举法求解得到银行最优收益F(X,Y,Z)。

\subsection{问题二的求解}
在问题一的基础上,对附件二中302家企业同样进行了数据清洗和处理,利用附件中的发票数据计算有效发票比,月平均交易利润,收益比等数值指标后,首先将月平均利润为负的企业筛去,不予贷款资格,此后和问题一的处理一样,基于TOPSIS算法求出各个企业的信誉风险指数。

由问题一已经得到不同信誉评级的企业的平均指标值,即平均信誉风险指数(见\cref{tab:16}),由此对附件二中302家企业作信誉评级的预测估计,信誉评级估计的范围标准为

\begin{table}[H]   %[H]
	\caption{信誉评级估计的范围标准}\label{tab:23} \centering
	\begin{tabular}{cc}
		\toprule[1.5pt]
		信誉评级 & 平均指标值$f_j^*$所在区间 \\
		\midrule[1pt]
		A & [0.6473,1)\\
		B & [0.6459,0.6473)\\
		C & [0.6446,0.6459)\\
		D & (0,0.6446)\\
		\bottomrule[1.5pt]
	\end{tabular}
\end{table}

对302家企业的信誉评级的部分结果见下表,完整的结果见支撑材料。
\begin{table}[H]   %[H]
	\caption{信誉评级的部分结果}\label{tab:24} \centering
	\begin{tabular}{ccccc}
		\toprule[1.5pt]
企业代号 & 收益比  & $f_i^*$   & $f_i^{**}$     & 评级 \\
		\midrule[1pt]
E389 & 162.7517569 & 0.647308502 & 0.632258502 & D  \\
E125 & 0.001653514 & 0.662474438 & 0.647424438 & A  \\
E126 & 3.817037047 & 0.662413765 & 0.647363765 & B  \\
E127 & 374.7612658 & 0.662340299 & 0.647290299 & B  \\
E128 & 22.74561681 & 0.662403311 & 0.647353311 & B  \\
E129 & 2.620565666 & 0.662413695 & 0.647363695 & B  \\
E130 & 0.410711597 & 0.662415509 & 0.647365509 & B  \\
E131 & 0.766691792 & 0.662414759 & 0.647364759 & B  \\
E132 & 2.064360408 & 0.662413615 & 0.647363615 & B \\
		\bottomrule[1.5pt]
\end{tabular}
\end{table}

根据评级结果,将信誉评级为D的企业和月平均利润为负的企业筛去,不予借贷资格。同问题一的第二个模型:贷款配额模型类似,继续运用TOPSIS算法,用四个指标计算出新综合指标值——称为贷款分配值(可贷款企业的评估指数)。

我们用\cref{eq:zhibiaobi}计算各个企业的贷款分配值比例,依据该比例,对这些拥有贷款资格的企业将银行的总贷款额(一亿元)进行第一次预分配。

当前待分配的总信贷款额为100000000元,根据余下可贷款的254家企业的贷款分配占比计算第一次的预分配额,其计算公式为:
\begin{equation}
W_j^1=\frac{f_j^*}{\sum_{j=1}^n f_j^*}
\end{equation}

\begin{equation}
R_j^1=100000000\times W_j^1
\end{equation}

由于企业平均年度成本额的限制,所以在第一次预分配后,有59家企业分配到的信贷额超过企业平均年度成本额,计算二者差值确定预分配额的剩余值为15132563.3,这部分分配额将作为第二次预分配信贷总额,同时这59家企业将完成信贷额分配。剩余的195企业分配到的信贷额低于企业年度平均成本额,计算得到第一次预分配额和企业年度成本额之间的差值,确定企业仍需要的信贷额。195家企业仍需要的信贷额部分结果见下表(其余详见支撑材料):

\begin{table}[H]   %[H]
	\caption{195家企业仍需要的信贷额部分结果}\label{tab:25} \centering
	\begin{tabular}{ccc}
		\toprule[1.5pt]
企业代号 & 评级 & 仍需信贷额       \\
		\midrule[1pt]
E417 & A  & 133547.9201 \\
E346 & A  & 518506.1784 \\
E396 & A  & 241188.683  \\
E355 & A  & 101197.7699 \\
E402 & A  & 93916.53855 \\
E188 & A  & 30277295.18 \\
E265 & A  & 9962641.136 \\
E322 & A  & 4125131.679 \\
E278 & A  & 4169021.672 \\
E287 & A  & 5484403.429 \\
\bottomrule[1.5pt]
\end{tabular}
\end{table}

接下来进行第二次预分配,由于已经有59家企业完成信贷分配,所以更新信贷分配占比的值,作为企业的第二次信贷分配占比。

在第二次预分配后,有3家企业完成信贷额分配,其企业代号分别为E161、E318和E368,第二次预分配额的剩余值为166925.4074,作为第三次预分配信贷总额的值。剩余的192企业仍需要的信贷额(详见支撑材料)。

经过两次的分配之后,共有62家企业完成信贷额分配,剩余的192家企业累积信贷分配额全部超过10万元。但对于已完成信贷额分配的62家企业而言,其中的23家企业的信贷分配额低于10万元,所以对这些企业不给予贷款,共计954276.8861,将这部分信贷额累加到第二次预分配额的剩余值中,作为第三次预分配信贷总额的值,共计1121202.294。

之后进行第三次预分配,更新企业的第三次信贷分配占比,计算第三次预分配额。结果显示所有企业的第三次预分配额均小于企业仍需要的信贷额,无剩余的待分配信贷额,因此这192家企业结束信贷额分配,其分配到的信贷额为第一、二、三次分配的信贷额总和。192家企业的分配到的信贷额总额部分结果见下表(其余详见支撑材料):

\begin{table}[H]   %[H]
	\caption{192家企业的分配到的信贷额总额部分结果}\label{tab:26} \centering
	\begin{tabular}{ccccc}
		\toprule[1.5pt]
企业代号 & 贷款分配额       & 二次分配额       & 三次分配额       & 总分信贷额       \\
		\midrule[1pt]
E417 & 388291.4599 & 77602.62034 & 5839.580615 & 471733.6609 \\
E346 & 388291.4616 & 77602.62067 & 5839.58064  & 471733.6629 \\
E396 & 388291.461  & 77602.62055 & 5839.580632 & 471733.6622 \\
E355 & 388291.4609 & 77602.62054 & 5839.58063  & 471733.6621 \\
E402 & 388291.4614 & 77602.62064 & 5839.580638 & 471733.6627 \\
E188 & 388291.4638 & 77602.62112 & 5839.580674 & 471733.6656 \\
E265 & 388291.4637 & 77602.62109 & 5839.580672 & 471733.6655 \\
E322 & 388291.4634 & 77602.62104 & 5839.580668 & 471733.6651 \\
E278 & 388291.4634 & 77602.62104 & 5839.580668 & 471733.6652 \\
E287 & 388291.4636 & 77602.62106 & 5839.58067  & 471733.6653 \\
\bottomrule[1.5pt]
\end{tabular}
\end{table}

至此,完成年度信贷总额的分配,其部分结果如下表所示(其余详见支撑材料):
\begin{table}[H]   %[H]
	\caption{年度信贷总额的分配结果}\label{tab:27} \centering
	\begin{tabular}{cccc}
		\toprule[1.5pt]
企业代号 & 信贷分配额 & 企业代号 & 信贷分配额\\
		\midrule[1pt]
E423 & 138103.4 &  E347 & 376309.1933 \\
E416 & 276890.61 & E348 & 287623.8035  \\
E411 & 151020.618 & E425 & 259040.96 \\
E397 & 379405.7143 & E393 & 151108.4133\\
E369 & 292966.9745 & E366 & 304049.048 \\
\bottomrule[1.5pt]
\end{tabular}
\end{table}

利用问题一的利率制定模型,目前已排除年利润为负的29家中小微企业,已知273家中小微企业的信誉评级,其中信誉评级为A、B、C、D企业的个数分别为:67,181,6,19,除去信誉等级为D的19家中小微企业不发放贷款,对剩余的254家企业按照信誉等级分为三类,分别计算信誉等级为A、B、C的企业贷款分配额:
\begin{table}[H]   %[H]
	\caption{信誉等级为A、B、C的企业贷款分配额}\label{tab:30} \centering
	\begin{tabular}{cc}
		\toprule[1.5pt]
		信誉等级 & 企业贷款分配额  \\
		\midrule[1pt]
		A    & 30438877.7 \\
		B    & 69427206.68 \\
		C    & 133915.6159 \\
		\bottomrule[1.5pt]
	\end{tabular}
\end{table}

第n年不同信誉等级的客户流失率分别为$x_n$,$y_n$和$z_n$,次年的各类企业总贷款额分别下降为$a_{n+1}=a_n(1-x_n)$,$b_{n+1}=b_n(1-y_n)$和$c_{n+1}=c_n(1-z_n)$。

由问题一的利率制定模型,X、Y、Z分别为第n年不同信誉等级的贷款年利率,其值与客户流失率$x_n$,$y_n$和$z_n$对应,由利率制定模型中的银行最优收益的目标函数:
\begin{equation}
F(X,Y,Z)=a_n(1-x_n)X+b_n(1-y_n)Y+c_n(1-z_n)Z
\end{equation}
通过穷举法求解得到银行最优收益F(X,Y,Z)为40601900,此时:

信誉等级为A的客户流失率$x_n$为0.135727183,对应的贷款年利率X为0.0465;

信誉等级为B的客户流失率$y_n$为0.302883401,对应的贷款年利率Y为0.0585;

信誉等级为C的客户流失率$z_n$为0.290189098,对应的贷款年利率Z为0.0585。

至此,得到不同信誉等级的企业贷款年利率,结果见下表:
\begin{table}[H]   %[H]
	\caption{不同信誉等级的企业贷款年利率}\label{tab:29} \centering
	\begin{tabular}{ccc}
	\toprule[1.5pt]
信誉等级 & 客户流失率       & 贷款年利率  \\
	\midrule[1pt]

A    & 0.135727183 & 0.0465 \\
B    & 0.302883401 & 0.0585 \\
C    & 0.290189098 & 0.0585 \\
\bottomrule[1.5pt]
\end{tabular}
\end{table}

\subsection{问题三的模型建立与求解}
对问题三,建立模糊综合评价模型求得以这302家企业为样本的不同类型企业的综合应对能力等级。

首先对302家企业进行粗略的分类,分为以下7类:
\begin{table}[H]   %[H]
	\caption{7种企业类型}	\centering
	\begin{tabular}{cc}
		\toprule[1.5pt]
		企业类型 & 企业个数\\
		\midrule[1pt]
		个体经营 & 56\\
		物流行业 & 11\\
		贸易/销售 & 35\\
		医疗/药物 & 10\\
		制造/生产 & 46\\
		服务行业 & 82\\
		网络/科技/文化 & 58\\
		\bottomrule[1.5pt]
	\end{tabular}
\end{table}	
不同企业类型应对突发因素的平均能力由查文献可得。我们对分类后的企业计算出月平均发票数、月平均利润,加上问题二所求得的信贷风险指数以及查阅资料所得的平均应对能力,将四者作为因素集$U$,因为

\begin{itemize}
	\item 月平均发票数体现企业的规模;
	\item 月平均利润体现企业的现金流大小;
	\item 信贷风险指数体现企业的综合实力;
	\item 平均应对能力则是衡量不同类型企业应对突发因素的关键指标。
\end{itemize}

\begin{table}[H]   %[H]
	\caption{指标体系集合}	\centering
	\begin{tabular}{cc}
		\toprule[1.5pt]
		符号 & 因素\\
		\midrule[1pt]
		$u_1$ & 月平均发票数\\
		$u_2$ & 月平均利润\\
		$u_3$ & 信贷风险指数\\
		$u_4$ & 平均应对能力\\
		\bottomrule[1.5pt]
	\end{tabular}
\end{table}	

\begin{equation}
U=(u_1,u_2,u_3,u_4)
\end{equation}

第二步,确定评语级。由于每个因素的评价值不同,往往会形成不同的等级。由各种不同决断构成的集合称为评语级,记为
\begin{align}
V & =(v_1,v_2,...,v_5)\\
  & =(\mbox{优秀}v_1,\mbox{良好}v_2,\mbox{一般}v_3,\mbox{较差}v_4,\mbox{差}v_5)
\end{align}

第三步,确定各因素的权重,它是$U$上的一个模糊向量,记为
\begin{equation}
\mathbf{Q}=(q_1,q_2,q_3,q_4)=(0.15,0.4,0.15,0.3)
\end{equation}

第四步,确定模糊综合判断矩阵。对指标$u_i$来说,对各个评语的隶属度为V上的模糊子集。对指标$u_i$的评判记为
\begin{equation}
R_i=(r_{i1},r_{i2},r_{i3},r_{i4},r_{i5})
\end{equation}
各指标的模糊综合判断矩阵为

\[
\mathbf{R} = \left(
\begin{array}{cccc}
r_{11} & r_{12} & \ldots & r_{15}\\
r_{21} & r_{22} & \ldots & r_{25}\\
\vdots & \vdots & \ddots & \vdots\\
r_{41} & r_{42} & \ldots & r_{45}\\
\end{array} \right)
\]
它是一个从$U$到$V$的模糊综合关系矩阵。

以制造业为例,以下为其模糊综合判断矩阵表(其余行业的模糊综合判断矩阵放入支撑材料):
\begin{table}[H]   %[H]
	\caption{制造行业的R矩阵}	\centering
	\begin{tabular}{cccccccc}
		\toprule[1.5pt]
行业   & 因素     & 因素具体值  & 优秀  & 良好  & 一般  & 较差  & 很差  \\
		\midrule[1pt]

制造   & 月均发票值  & 20.93453183 & 0.2 & 0.1 & 0.5 & 0.1 & 0.1 \\
制造   & 平均应对能力 & 6           & 0.3 & 0.3 & 0.3 & 0.1 & 0    \\
制造   & 月均利    & 477820.3643 & 0.3 & 0.3 & 0.3 & 0.1 & 0    \\
制造   & 信贷风险   & 0.026433484 & 0.2 & 0.3 & 0.3 & 0.1 & 0.1  \\
		\bottomrule[1.5pt]
\end{tabular}
\end{table}	

最后是综合评判。利用R就可以得到一个模糊变换
\begin{equation}
T_R:F(U)\rightarrow F(V)
\end{equation}
那么,由$$\mathbf{B}=\mathbf{Q}\times\mathbf{R}$$即可得到综合评判等级,见下表

\begin{table}[H]   %[H]
	\caption{不同类型企业的综合评判等级的结果B}	\centering
	\begin{tabular}{cccccc}
		\toprule[1.5pt]
		企业类型 & & & 各评级占比& & \\
		\midrule[1pt]
     & 优秀    & 良好    & 一般    & 较差    & 很差    \\
制造/生产   & 0.255 & 0.27  & 0.33  & 0.1   & 0.045 \\
贸易/销售   & 0.2   & 0.2   & 0.37  & 0.16  & 0.07  \\
医疗/药物   & 0.395 & 0.39  & 0.2   & 0.015 & 0     \\
物流行业   & 0.37  & 0.375 & 0.155 & 0.07  & 0     \\
网络/科技/文化 & 0.415 & 0.385 & 0.13  & 0.055 & 0     \\
服务行业   & 0.045 & 0.17  & 0.215 & 0.36  & 0.24  \\
个体经营 & 0.07  & 0.13  & 0.24  & 0.37  & 0.19 \\
		\bottomrule[1.5pt]
\end{tabular}
\end{table}	
最后得到不同类型企业的综合应对能力等级结果如下:

\begin{table}[H]   %[H]
	\caption{不同类型企业的综合评判等级}	\centering
	\begin{tabular}{cc}
		\toprule[1.5pt]
		企业类型 & 综合评判等级\\
		\midrule[1pt]
		制造/生产   & 一般\\
		贸易/销售   & 一般\\
		医疗/药物   & 优秀\\
		物流行业   & 良好\\
		网络/科技/文化 & 优秀 \\
		服务行业   & 较差\\
		个体经营 & 较差 \\
		\bottomrule[1.5pt]
	\end{tabular}
\end{table}	

在完成不同类型企业综合应对能力的模糊评价后,将评价等级为“较差”的企业挑选出来,减少对这部分企业的已有信贷额分配。根据查阅的资料得,降低的程度为这部分企业已有信贷额的10$\%$,减少的信贷额总量为5176048.014,同时调整这部分企业的信贷额分配为原有信贷额的90$\%$(数据详见支撑材料)。

接下来将“较差”企业减少的信贷额总量依次分配给评价等级为“优秀”,“较好”和“一般”的企业。根据“优秀”企业已知的信贷风险值计算企业的信贷额分配占比,从而得到各企业第一次的预分配信贷额,通过计算各企业已分配的信贷额和第一次预分配的信贷额的总和在100000到1000000之间,满足银行的可贷款额度,因此结束信贷额分配的调整,得到“优秀”企业最终分配到的信贷额(数据详见支撑材料)。
由于在“优秀”企业中结束了信贷额分配的调整,因此评价等级为“较好”和“一般”的企业的信贷分配额将不再调整。并且贷款年利率与第二题相同,最终调整后的信贷额分配部分见下表(数据结果详见支撑材料)。

\begin{table}[H]   %[H]
	\caption{各类型企业调整分配额的部分结果}	\centering
	\begin{tabular}{ccc}
		\toprule[1.5pt]
		企业类型 & 综合评判等级\\
		\midrule[1pt]
企业代号 & 信誉评级 & 总分配信贷额      \\
E125 & A    & 424560.2992 \\
E126 & B    & 424560.2994 \\
E127 & B    & 424726.3183 \\
E128 & B    & 424560.333  \\
E129 & B    & 424560.2994 \\
E130 & B    & 424560.2991 \\
E131 & B    & 424560.2992 \\
E132 & B    & 424560.2993 \\
E133 & B    & 424560.2994 \\
E135 & B    & 424560.3008 \\
		\bottomrule[1.5pt]
\end{tabular}
\end{table}	

\section{模型评价}
\subsection{问题一和二模型的评价}
\textbf{模型的优点:}
\begin{itemize}
	\item 对数据分布、样本含量指标大小均无严格的限制,适用于多评价单元、多指标的大系统资料,例如本题中的中小微企业进项、销项发票数据;
	\item 既可以用于横向对比,又可以用于纵向分析,应用十分灵活,并且数学计算相对比较简单,其结果量化客观;
	\item 比较充分地利用了原有的数据信息,与实际情况较为吻合;
	\item 可对每个评价对象的优劣进行排序。
\end{itemize}

\textbf{模型的缺点:}
\begin{itemize}
	\item 当两个评价对象的指标值关于最优方案和最劣方案的连线对称时,无法得出准确的结果;
	\item 只能对每个评价对象的优劣进行排序,不能分档管理,灵敏度不高;
	\item 权重$w_j(j=1,2,…,n)$不是通过计算得到,而是事先确定的,其值一般是主观赋值,所以具有一定的随意性。
\end{itemize}

\textbf{模型的改进:}
可以尝试对各个指标进行进行相关性分析,从而确定决策的权重。

\subsection{问题三模型的评价}
\textbf{模型的优点:}
\begin{itemize}
	\item 可以将不完全信息转化为模糊概念,使定性问题定量化,提高评估的准确性、可信性;
\end{itemize}

\textbf{模型的缺点:}
\begin{itemize}
	\item 只考虑了主要因素的作用,忽视了次要因素,使评价结果不够全面;
	\item 当指标数较多时,权向量与模糊矩阵R不匹配,易造成评判失败。
	\item 评价的主观性明显。
	
\end{itemize}

\textbf{模型的改进:}
为避免模糊数学中隶属函数构造的随意性,在模糊综合评价法中引入物元分析法的关联度函数,改进模糊综合评价法的隶属函数,建立并使用了模糊物元综合评价模型。


%参考文献
\begin{thebibliography}{9}%宽度9
    \bibitem{1} \url{https://www.cnblogs.com/pxlsdz/p/12364711.html}
    \bibitem{2}古莹奎,付阳,梁玲强,承姿辛. 基于TOPSIS理论的FMEA灵敏度分析方法[J]. 煤炭学报,2016,(S2).
    \bibitem{3}何正柯. TOPSIS多属性决策方法的改进研究[D]. 辽宁科技大学,2018.
    \bibitem{4}周亚. 多属性决策中的TOPSIS法研究[D]. 武汉理工大学,2009.
    \bibitem{5} \url{https://zhuanlan.zhihu.com/p/52100246}
    \bibitem{6}张烨. 中小微企业关系型贷款业务研究[D]. 华东理工大学: 华东理工大学,2014.
    \bibitem{7}吴矜. 中小企业关系型贷款违约风险研究[D]. 中南林业科技大学: 中南林业科技大学,2018.
\end{thebibliography}


\newpage

\begin{appendices}

\section{问题一的信誉风险指数模型的决策矩阵}

\[
\mathbf{A} = \left(
\begin{array}{ccc}
0.947788005 & 11449458.14  & 2.677610195  \\
0.910869294 & 1733962.083  & 0.199753671  \\
0.975488992 & 11927113.1   & 7.138390843  \\
0.926608553 & 5170415.261  & 1.152900397  \\
0.976691633 & 7764645.258  & 11.20460106  \\
0.917290061 & 3169112.982  & 1.162096237  \\
0.968490879 & 10379294.01  & 43.21154892  \\
0.942467206 & 8971333.137  & 796.775134   \\
0.886862076 & 662710.9032  & 0.171732432  \\
0.925869522 & 1616222.432  & 0.434112511  \\
0.909309613 & 2990098.009  & 1.477278227  \\
0.949092784 & 1762979.644  & 0.756462257  \\
0.969777458 & 1158661.639  & 8.473754038  \\
0.949541284 & 708784.54    & 2228.314044  \\
0.962319487 & 985674.8545  & 2.605194062  \\
0.932086401 & 688524.2666  & 3.204821181  \\
0.965438663 & 227606.6478  & 0.397501899  \\
0.966545455 & 463066.1347  & 97.07451626  \\
\end{array} \right)
\]

\section{问题一的各个信誉评级企业相关实力数据}
\begin{table}[H]   %[H]
	\caption{附件一中信誉评级为A的企业相关实力数据}\label{tab:001} \centering
	\begin{tabular}{cccc}
		\toprule[1.5pt]
		企业 & 平均有效发票比     & 月均利/元          & 收益比          \\
		\midrule[1pt]
		E2  & 0.947788005 & 11449458.14  & 2.677610195  \\
		E6  & 0.910869294 & 1733962.083  & 0.199753671  \\
		E7  & 0.975488992 & 11927113.1   & 7.138390843  \\
		E8  & 0.926608553 & 5170415.261  & 1.152900397  \\
		E9  & 0.976691633 & 7764645.258  & 11.20460106  \\
		E13 & 0.917290061 & 3169112.982  & 1.162096237  \\
		\bottomrule[1.5pt]
\end{tabular}
\end{table}
\begin{table}[H]   %[H]
	\caption{附件一中信誉评级为A的企业相关实力数据(续表)} \centering
	\begin{tabular}{cccc}
		\toprule[1.5pt]
		企业 & 平均有效发票比     & 月均利/元          & 收益比          \\
		\midrule[1pt]

		E15 & 0.968490879 & 10379294.01  & 43.21154892  \\
		E16 & 0.942467206 & 8971333.137  & 796.775134   \\
		E17 & 0.886862076 & 662710.9032  & 0.171732432  \\
		E18 & 0.925869522 & 1616222.432  & 0.434112511  \\
		E22 & 0.909309613 & 2990098.009  & 1.477278227  \\
		E24 & 0.949092784 & 1762979.644  & 0.756462257  \\
		E31 & 0.969777458 & 1158661.639  & 8.473754038  \\
		E42 & 0.949541284 & 708784.54    & 2228.314044  \\
		E48 & 0.962319487 & 985674.8545  & 2.605194062  \\
		E54 & 0.932086401 & 688524.2666  & 3.204821181  \\
		E59 & 0.965438663 & 227606.6478  & 0.397501899  \\
		E64 & 0.966545455 & 463066.1347  & 97.07451626  \\
		E81 & 0.95496535  & 38356.58405  & 0.62923569   \\
		E84 & 0.937527889 & 188683.9972  & 2.047512794  \\
		E88 & 0.955       & 50745.50838  & 2.747180582  \\
		E91 & 0.951320457 & 27973.89703  & 0.692819173 \\
		\bottomrule[1.5pt]
	\end{tabular}
\end{table}

\begin{table}[H]   %[H]B
	\caption{附件一中信誉评级为B的企业相关实力数据}\label{tab:003} \centering
	\begin{tabular}{cccc}
		\toprule[1.5pt]
		企业 & 平均有效发票比     & 月均利/元           & 收益比          \\
		\midrule[1pt]
		E5   & 0.954462321 & 18198.08895  & 0.002980076  \\
		E10  & 0.925301749 & 9562118.778  & 51.65832313  \\
		E12  & 0.950546048 & 3231195.703  & 1.074908497  \\
		E23  & 0.966885308 & 211097.4695  & 0.038496672  \\
		E28  & 0.922874138 & 1906159.489  & 31.05832784  \\
		E30  & 0.95759118  & 1542893.315  & 1.94901473   \\
		E32  & 0.921259116 & 1346289.596  & 16.00351781  \\
		E34  & 0.869961831 & 832742.1335  & 0.483296226  \\
		E35  & 0.951665599 & 35338.1527   & 0.034686224  \\
		E37  & 0.944007606 & 32443388.63  & 23.01714992  \\
		E38  & 0.922840597 & 1081673.117  & 2.10048715   \\
		E43  & 0.87263213  & 1007686.754  & 2.881145479  \\
		E45  & 0.962564103 & 132874.3921  & 0.141850113  \\
		E51  & 0.963714282 & 188658.3997  & 0.31508998   \\
		E57  & 0.909870039 & 92917.585    & 0.089714035  \\
		E58  & 0.851556305 & 515167.3983  & 1.842372883  \\
		E60  & 0.905820416 & 108613.6566  & 0.573156262  \\
		E61  & 0.98363738  & 486064.5348  & 17.01111021  \\
		E62  & 0.933242871 & 223314.6073  & 6.357877669  \\
		E63  & 0.949399176 & 285789.0313  & 0.270339268  \\
		E65  & 0.962257855 & 80196.60912  & 0.524451693  \\
		E67  & 0.931055677 & 113564.3539  & 3.135289431  \\
		\bottomrule[1.5pt]
\end{tabular}
\end{table}
		
\begin{table}[H]   %[H]B
	\caption{附件一中信誉评级为B的企业相关实力数据(续表)}\label{tab:002} \centering
	\begin{tabular}{cccc}
		\toprule[1.5pt]
		企业 & 平均有效发票比     & 月均利/元           & 收益比          \\
		\midrule[1pt]		
		E70  & 0.941221019 & 102737.1066  & 0.552722151  \\
		E71  & 0.957221075 & 11271.38324  & 0.034950816  \\
		E74  & 0.961261261 & 456223.755   & 10.82012374  \\
		E76  & 0.949387755 & 85309.29514  & 3.366835411  \\
		E79  & 0.923739052 & 9565.553158  & 0.109338609  \\
		E85  & 0.9193183   & 72496.57355  & 5.154437881  \\
		E93  & 0.964614262 & 39957.02793  & 3.454977078  \\
		E95  & 0.973165389 & 577420.92    & 2029.599016  \\
		E97  & 0.982954545 & 95264.612    & 47.91982495  \\
		E98  & 0.924836601 & 63354.11684  & 5.746021676  \\
		E106 & 0.937091503 & 30998.57235  & 8.574134371 \\
		\bottomrule[1.5pt]
	\end{tabular}
\end{table}

\begin{table}[H]   %[H]B
	\caption{附件一中信誉评级为C的企业相关实力数据}\label{tab:004} \centering
	\begin{tabular}{cccc}
		\toprule[1.5pt]
		企业 & 平均有效发票比     & 月均利/元           & 收益比          \\
		\midrule[1pt]
E3   & 0.970736223 & 13635553.07 & 9.563625117  \\
E4   & 0.924264076 & 46288359.59 & 6.363680435  \\
E11  & 0.949123717 & 661743.1387 & 0.105539745  \\
E14  & 0.925320384 & 3717694.856 & 1.040091947  \\
E25  & 0.909417831 & 853573.735  & 2.900935535  \\
E29  & 0.973684211 & 3000723.737 & 241.1918996  \\
E39  & 0.941650548 & 1200872.85  & 77.21817833  \\
		\bottomrule[1.5pt]
\end{tabular}
\end{table}

\begin{table}[H]   %[H]B
	\caption{附件一中信誉评级为C的企业相关实力数据(续表)}\label{tab:008} \centering
	\begin{tabular}{cccc}
		\toprule[1.5pt]
		企业 & 平均有效发票比     & 月均利/元           & 收益比          \\
		\midrule[1pt]
E40  & 0.940294652 & 841735.3089 & 6.566770748  \\
E41  & 0.9402732   & 742248.9689 & 5.096418907  \\
E44  & 0.949952943 & 82354.94639 & 0.086231454  \\
E46  & 0.946145377 & 249042.2831 & 0.170112323  \\
E47  & 0.960630062 & 229497.0942 & 0.172374722  \\
E49  & 0.935905128 & 469583.6769 & 0.555817133  \\
E50  & 0.919307197 & 1339573.411 & 9.263025907  \\
E53  & 0.912313824 & 107421.2984 & 0.095298944  \\
E55  & 0.832288856 & 472482.8318 & 1.218101607  \\
E56  & 0.942226596 & 50486.02735 & 0.0609481    \\
E68  & 0.945205479 & 488892.5925 & 7927.987986  \\
E69  & 0.94548552  & 213886.0435 & 1118.788535  \\
E72  & 0.842018197 & 163621.5711 & 1.903788081  \\
E73  & 0.967507132 & 149006.7389 & 6.266964053  \\
E75  & 0.953176726 & 108186.5703 & 1.431131706  \\
E77  & 0.979112834 & 240627.8413 & 4.467759014  \\
E78  & 0.962686567 & 200980.9163 & 1.608289666  \\
E80  & 0.983254219 & 143846.335  & 0.44248756   \\
E86  & 0.928194993 & 33261.26154 & 0.248251378  \\
E87  & 0.96917564  & 107656.3777 & 1.388077425  \\
E90  & 0.843849059 & 78687.41031 & 3.322822789  \\
E92  & 0.959191044 & 171738.1258 & 12.92945825  \\
		\bottomrule[1.5pt]
\end{tabular}
\end{table}

\begin{table}[H]   %[H]B
	\caption{附件一中信誉评级为C的企业相关实力数据(续表)}\label{tab:009} \centering
	\begin{tabular}{cccc}
		\toprule[1.5pt]
		企业 & 平均有效发票比     & 月均利/元           & 收益比          \\
		\midrule[1pt]
E94  & 0.954081633 & 43297.92542 & 63.57214058  \\
E104 & 0.952380952 & 26238.636   & 937.0941429  \\
E105 & 0.995901639 & 39886.19348 & 108.1818927  \\
E110 & 0.873493976 & 9331.260476 & 188.4196827 \\
		\bottomrule[1.5pt]
\end{tabular}
\end{table}

\begin{table}[H]   %[H]B
	\caption{附件一中信誉评级为D的企业相关实力数据}\label{tab:005} \centering
	\begin{tabular}{cccc}
		\toprule[1.5pt]
		企业 & 平均有效发票比     & 月均利/元           & 收益比          \\
		\midrule[1pt]
E36  & 0.939245531 & 489143.0597  & 0.012176318  \\
E52  & 0.858485178 & 99209.2008   & 0.004664254  \\
E82  & 0.981988986 & 56019.99615  & 0.016470287  \\
E100 & 0.994949495 & 22423.44586  & 0.03242776   \\
E101 & 0.753968254 & 73660.9525   & 0.187729568  \\
E103 & 0.91315407  & 63732.635    & 0.034970821  \\
E107 & 0.814814815 & 62014.09182  & 0.074357424  \\
E108 & 1           & 23043.5275   & 0.116844546  \\
E109 & 0.848484848 & 36625.7775   & 0.080177794  \\
E111 & 0.882820144 & 7782.067333  & 0.008851891  \\
E112 & 0.844768439 & 14557.159    & 0.013296601  \\
E115 & 0.833333333 & 13751.42     & 0.240409441  \\
E116 & 0.957446809 & 15254.778    & 0.06399613   \\
E117 & 0.795454545 & 51664.252    & 0.098376619  \\
		\bottomrule[1.5pt]
\end{tabular}
\end{table}

\begin{table}[H]   %[H]B
	\caption{附件一中信誉评级为D的企业相关实力数据(续表)} \centering
	\begin{tabular}{cccc}
		\toprule[1.5pt]
		企业 & 平均有效发票比     & 月均利/元           & 收益比          \\
		\midrule[1pt]
E118 & 0.937739464 & 7017.836333  & 0.024979734  \\
E120 & 0.641283525 & 28759.60833  & 0.146438757  \\
E122 & 0.921786723 & 317.2965385  & 0.005223797  \\
E123 & 0.753846154 & 13296.22667  & 0.058429619 \\
		\bottomrule[1.5pt]
\end{tabular}
\end{table}

\section{问题一的允许贷款企业的距离值以及综合指标值}

\begin{table}[H]   %[H]
	\caption{附件一中允许贷款的信誉为B的企业的距离值以及综合指标值}\label{tab:011} \centering
	\begin{tabular}{cccc}
		\toprule[1.5pt]
		企业 & $s_i^*$    & $s_i^0$        & $f_i^*$         \\
		\midrule[1pt]
E5 & 0.000463328 & 0.000881661 & 0.655515425 \\ 
E10 & 0.000463328 & 0.000867527 & 0.651856918 \\ 
E12 & 0.000463328 & 0.000868817 & 0.65219402 \\ 
E23 & 0.000463328 & 0.000869904 & 0.652477562 \\ 
E28 & 0.000463328 & 0.000864957 & 0.651183139 \\ 
E30 & 0.000463328 & 0.000868669 & 0.652155384 \\ 
E32 & 0.000463328 & 0.000866084 & 0.651478886 \\ 
E34 & 0.000463328 & 0.000868941 & 0.652226175 \\ 
E35 & 0.000463328 & 0.00087519 & 0.653849848 \\ 
E37 & 0.000463328 & 0.00086866 & 0.652152967 \\ 
E38 & 0.000463328 & 0.00086856 & 0.652126845 \\ 
E43 & 0.000463328 & 0.000868339 & 0.652068976 \\ 
E45 & 0.000463328 & 0.00087034 & 0.652591144 \\ 
E51 & 0.000463328 & 0.000869669 & 0.652416248 \\ 

		\bottomrule[1.5pt]
\end{tabular}
\end{table}

\begin{table}[H]   %[H]
	\caption{附件一中允许贷款的信誉为B的企业的距离值以及综合指标值(续表)}\label{tab:014} \centering
	\begin{tabular}{cccc}
		\toprule[1.5pt]
		企业 & $s_i^*$    & $s_i^0$        & $f_i^*$         \\
		\midrule[1pt]
E57 & 0.000463328 & 0.00087099 & 0.652760262 \\ 
E58 & 0.000463328 & 0.000868356 & 0.652073449 \\ 
E60 & 0.000463328 & 0.000869577 & 0.652392272 \\ 
E61 & 0.000463328 & 0.000860761 & 0.650077754 \\ 
E62 & 0.000463328 & 0.000862883 & 0.650637577 \\ 
E63 & 0.000463328 & 0.000869409 & 0.652348387 \\ 
E65 & 0.000463328 & 0.000870164 & 0.652545339 \\ 
E67 & 0.000463328 & 0.000864076 & 0.650951737 \\ 
E70 & 0.000463328 & 0.000869753 & 0.652438188 \\ 
E71 & 0.000463328 & 0.000888937 & 0.657368729 \\ 
E74 & 0.000463328 & 0.000863538 & 0.650810212 \\ 
E76 & 0.000463328 & 0.000861894 & 0.65037686 \\ 
E79 & 0.000463328 & 0.000889754 & 0.657575767 \\ 
E85 & 0.000463328 & 0.000854552 & 0.648429142 \\ 
E93 & 0.000463328 & 0.000853599 & 0.648174841 \\ 
E95 & 0.000463328 & 0.000467523 & 0.50225312 \\ 
E97 & 0.000463328 & 0.000750957 & 0.618435351 \\ 
E98 & 0.000463328 & 0.000850245 & 0.647276522 \\ 
E106 & 0.000463328 & 0.000809129 & 0.635879272 \\ 
		\bottomrule[1.5pt]
\end{tabular}
\end{table}

\begin{table}[H]   %[H]
	\caption{附件一中允许贷款的信誉为C的企业的距离值以及综合指标值}\label{tab:012} \centering
	\begin{tabular}{cccc}
		\toprule[1.5pt]
		企业 & $s_i^*$    & $s_i^0$        & $f_i^*$         \\
		\midrule[1pt]
E3 & 0.004670603 & 0.00905211 & 0.659644358 \\ 
E4 & 0.004670603 & 0.009052236 & 0.659647488 \\ 
E11 & 0.004670603 & 0.009052579 & 0.659655987 \\ 
E14 & 0.004670603 & 0.009052258 & 0.659648016 \\ 
E25 & 0.004670603 & 0.009051691 & 0.659633961 \\ 
E29 & 0.004670603 & 0.009032565 & 0.659158903 \\ 
E39 & 0.004670603 & 0.009036634 & 0.659260069 \\ 
E40 & 0.004670603 & 0.00905062 & 0.659607394 \\ 
E41 & 0.004670603 & 0.009050887 & 0.659614018 \\ 
E44 & 0.004670603 & 0.009054847 & 0.659712222 \\ 
E46 & 0.004670603 & 0.009053032 & 0.659667227 \\ 
E47 & 0.004670603 & 0.009053111 & 0.659669172 \\ 
E49 & 0.004670603 & 0.009052464 & 0.659653146 \\ 
E50 & 0.004670603 & 0.009050732 & 0.659610176 \\ 
E53 & 0.004670603 & 0.009054137 & 0.659694632 \\ 
E55 & 0.004670603 & 0.009052064 & 0.65964322 \\ 
E56 & 0.004670603 & 0.009056562 & 0.659754745 \\ 
E68 & 0.004670603 & 0.005435777 & 0.537856002 \\ 
E69 & 0.004670603 & 0.007790083 & 0.625172907 \\ 
E72 & 0.004670603 & 0.009050668 & 0.659608585 \\ 
E73 & 0.004670603 & 0.009043512 & 0.659430967 \\ 
E75 & 0.004670603 & 0.009051178 & 0.659621232 \\ 
E77 & 0.004670603 & 0.009048697 & 0.659559676 \\ 
		\bottomrule[1.5pt]
\end{tabular}
\end{table}

\begin{table}[H]   %[H]
	\caption{附件一中允许贷款的信誉为C的企业的距离值以及综合指标值(续表)}\label{tab:015} \centering
	\begin{tabular}{cccc}
		\toprule[1.5pt]
		企业 & $s_i^*$    & $s_i^0$        & $f_i^*$         \\
		\midrule[1pt]
E78 & 0.004670603 & 0.009051475 & 0.659628593 \\ 
E80 & 0.004670603 & 0.009053191 & 0.659671154 \\ 
E86 & 0.004670603 & 0.009057298 & 0.659772969 \\ 
E87 & 0.004670603 & 0.009051307 & 0.659624449 \\ 
E90 & 0.004670603 & 0.009044511 & 0.659455784 \\ 
E92 & 0.004670603 & 0.009035112 & 0.659222245 \\ 
E94 & 0.004670603 & 0.008697893 & 0.650626167 \\ 
E104 & 0.004670603 & 0.004673335 & 0.500146229 \\ 
E105 & 0.004670603 & 0.008396617 & 0.642571041 \\ 
E110 & 0.004670603 & 0.004801862 & 0.506928464 \\ 
		\bottomrule[1.5pt]
	\end{tabular}
\end{table}

\begin{table}[H]   %[H]
	\caption{附件一中允许贷款的信誉为D的企业的距离值以及综合指标值}\label{tab:013} \centering
	\begin{tabular}{cccc}
		\toprule[1.5pt]
		企业 & $s_i^*$    & $s_i^0$        & $f_i^*$         \\
		\midrule[1pt]
E36 & 0.000712473 & 0.000375579 & 0.34518449 \\ 
E52 & 0.000712473 & 0.000374711 & 0.344661994 \\ 
E82 & 0.000712473 & 0.000373606 & 0.343995456 \\ 
E100 & 0.000712473 & 0.000370391 & 0.342047309 \\ 
E101 & 0.000712473 & 0.000374829 & 0.344733345 \\ 
E103 & 0.000712473 & 0.000374047 & 0.344261697 \\ 
E107 & 0.000712473 & 0.000374283 & 0.34440376 \\ 
E108 & 0.000712473 & 0.000370958 & 0.342392015 \\ 
E109 & 0.000712473 & 0.000373131 & 0.343708281 \\ 
E111 & 0.000712473 & 0.00036227 & 0.337075867 \\ 
E112 & 0.000712473 & 0.000368661 & 0.340995018 \\ 
E115 & 0.000712473 & 0.000370367 & 0.342032869 \\ 
E116 & 0.000712473 & 0.000368487 & 0.340888424 \\ 
E117 & 0.000712473 & 0.000374084 & 0.344283826 \\ 
E118 & 0.000712473 & 0.000360329 & 0.335876329 \\ 
E120 & 0.000712473 & 0.00037361 & 0.34399768 \\ 
E122 & 0.000712473 & 0.000712473 & 0.5 \\ 
E123 & 0.000712473 & 0.000369244 & 0.341350107 \\ 
		\bottomrule[1.5pt]
	\end{tabular}
\end{table}

\section{附件一中允许贷款企业在贷款配额模型中距离值以及综合指标值以及综合指标值占比}

\begin{table}[H]   %[H]
\centering
	\begin{tabular}{cccccc}
		\toprule[1.5pt]
       企业& 信誉评级 & $s_j^*$ & $s_j^0$ & $f_j^*$ & $f_j^*/(\sum_{j=1}^n f_j^*)$ \\ 
		\midrule[1pt]

E2 & A & 1.95303243 & 0.027473055 & 0.013871739 & 0.004259526 \\ 
E3 & C & 1.95303243 & 0.027473055 & 0.013871739 & 0.004259526 \\ 
E4 & C & 1.95303243 & 0.027473055 & 0.013871739 & 0.004259526 \\ 
E5 & B & 1.95303243 & 0.027473055 & 0.013871739 & 0.004259526 \\ 
E6 & A & 1.95303243 & 0.027473055 & 0.013871739 & 0.004259526 \\ 
E7 & A & 1.95303243 & 0.027473055 & 0.013871739 & 0.004259526 \\ 
E8 & A & 1.95303243 & 0.027473055 & 0.013871739 & 0.004259526 \\ 
E9 & A & 1.95303243 & 0.027473056 & 0.013871739 & 0.004259526 \\ 
E10 & B & 1.95303243 & 0.027473073 & 0.013871748 & 0.004259528 \\ 
E11 & C & 1.95303243 & 0.027473055 & 0.013871739 & 0.004259526 \\ 
E12 & B & 1.95303243 & 0.027473055 & 0.013871739 & 0.004259526 \\ 
E13 & A & 1.95303243 & 0.027473055 & 0.013871739 & 0.004259526 \\ 
E14 & C & 1.95303243 & 0.027473055 & 0.013871739 & 0.004259526 \\ 
E15 & A & 1.95303243 & 0.027473063 & 0.013871743 & 0.004259527 \\ 
E16 & A & 1.95303243 & 0.028542729 & 0.014404061 & 0.004422983 \\ 
E17 & A & 1.95303243 & 0.027473055 & 0.013871739 & 0.004259526 \\ 
E18 & A & 1.95303243 & 0.027473055 & 0.013871739 & 0.004259526 \\ 
E22 & A & 1.95303243 & 0.027473055 & 0.013871739 & 0.004259526 \\ 
E23 & B & 1.95303243 & 0.027473055 & 0.013871739 & 0.004259526 \\ 
E24 & A & 1.95303243 & 0.027473055 & 0.013871739 & 0.004259526 \\ 
E25 & C & 1.95303243 & 0.027473055 & 0.013871739 & 0.004259526 \\ 
		\bottomrule[1.5pt]
\end{tabular}
\end{table}

\begin{table}[H]   %[H]
	\centering
	\begin{tabular}{cccccc}
		\toprule[1.5pt]
		企业& 信誉评级 & $s_j^*$ & $s_j^0$ & $f_j^*$ & $f_j^*/(\sum_{j=1}^n f_j^*)$ \\ 
		\midrule[1pt]
E28 & B & 1.95303243 & 0.02747311 & 0.013871766 & 0.004259534 \\ 
E29 & C & 1.95303243 & 0.02769776 & 0.013983611 & 0.004293878 \\ 
E30 & B & 1.95303243 & 0.027473055 & 0.013871739 & 0.004259526 \\ 
E31 & A & 1.95303243 & 0.027473056 & 0.01387174 & 0.004259526 \\ 
E32 & B & 1.95303243 & 0.027473063 & 0.013871743 & 0.004259527 \\ 
E34 & B & 1.95303243 & 0.027473055 & 0.013871739 & 0.004259526 \\ 
E35 & B & 1.95303243 & 0.027473055 & 0.013871739 & 0.004259526 \\ 
E37 & B & 1.95303243 & 0.027473056 & 0.013871739 & 0.004259526 \\ 
E38 & B & 1.95303243 & 0.027473055 & 0.013871739 & 0.004259526 \\ 
E39 & C & 1.95303243 & 0.027479225 & 0.013874811 & 0.004260469 \\ 
E40 & C & 1.95303243 & 0.027473056 & 0.013871739 & 0.004259526 \\ 
E41 & C & 1.95303243 & 0.027473056 & 0.013871739 & 0.004259526 \\ 
E42 & A & 1.95303243 & 0.86736561 & 0.307533049 & 0.094432641 \\ 
E43 & B & 1.95303243 & 0.027473055 & 0.013871739 & 0.004259526 \\ 
E44 & C & 1.95303243 & 0.027473055 & 0.013871739 & 0.004259526 \\ 
E45 & B & 1.95303243 & 0.027473055 & 0.013871739 & 0.004259526 \\ 
E46 & C & 1.95303243 & 0.027473055 & 0.013871739 & 0.004259526 \\ 
E47 & C & 1.95303243 & 0.027473055 & 0.013871739 & 0.004259526 \\ 
E48 & A & 1.95303243 & 0.027473055 & 0.013871739 & 0.004259526 \\ 
E49 & C & 1.95303243 & 0.027473055 & 0.013871739 & 0.004259526 \\ 
E50 & C & 1.95303243 & 0.027473059 & 0.013871741 & 0.004259526 \\ 
E51 & B & 1.95303243 & 0.027473055 & 0.013871739 & 0.004259526 \\ 
E53 & C & 1.95303243 & 0.027473055 & 0.013871739 & 0.004259526 \\ 
E54 & A & 1.95303243 & 0.027473056 & 0.013871739 & 0.004259526 \\ 
		\bottomrule[1.5pt]
\end{tabular}
\end{table}

\begin{table}[H]   %[H]
\centering
\begin{tabular}{cccccc}
\toprule[1.5pt]
企业& 信誉评级 & $s_j^*$ & $s_j^0$ & $f_j^*$ & $f_j^*/(\sum_{j=1}^n f_j^*)$ \\ 
\midrule[1pt]
E55 & C & 1.95303243 & 0.027473055 & 0.013871739 & 0.004259526 \\ 
E56 & C & 1.95303243 & 0.027473055 & 0.013871739 & 0.004259526 \\ 
E57 & B & 1.95303243 & 0.027473055 & 0.013871739 & 0.004259526 \\ 
E58 & B & 1.95303243 & 0.027473055 & 0.013871739 & 0.004259526 \\ 
E59 & A & 1.95303243 & 0.027473055 & 0.013871739 & 0.004259526 \\ 
E60 & B & 1.95303243 & 0.027473055 & 0.013871739 & 0.004259526 \\ 
E61 & B & 1.95303243 & 0.027473127 & 0.013871775 & 0.004259537 \\ 
E62 & B & 1.95303243 & 0.027473062 & 0.013871742 & 0.004259527 \\ 
E63 & B & 1.95303243 & 0.027473055 & 0.013871739 & 0.004259526 \\ 
E64 & A & 1.95303243 & 0.02756148 & 0.013915765 & 0.004273045 \\ 
E65 & B & 1.95303243 & 0.027473055 & 0.013871739 & 0.004259526 \\ 
E67 & B & 1.95303243 & 0.027473057 & 0.01387174 & 0.004259526 \\ 
E68 & C & 1.95303243 & 1.951887877 & 0.499853447 & 0.153487507 \\ 
E69 & C & 1.95303243 & 0.727716795 & 0.271460228 & 0.083355939 \\ 
E70 & B & 1.95303243 & 0.027473055 & 0.013871739 & 0.004259526 \\ 
E71 & B & 1.95303243 & 0.027473055 & 0.013871739 & 0.004259526 \\ 
E72 & C & 1.95303243 & 0.027473056 & 0.013871739 & 0.004259526 \\ 
E73 & C & 1.95303243 & 0.027473068 & 0.013871745 & 0.004259528 \\ 
E74 & B & 1.95303243 & 0.027473069 & 0.013871746 & 0.004259528 \\ 
E75 & C & 1.95303243 & 0.027473055 & 0.013871739 & 0.004259526 \\ 
E76 & B & 1.95303243 & 0.027473058 & 0.01387174 & 0.004259526 \\ 
E77 & C & 1.95303243 & 0.027473059 & 0.013871741 & 0.004259526 \\ 
E78 & C & 1.95303243 & 0.027473055 & 0.013871739 & 0.004259526 \\ 
E79 & B & 1.95303243 & 0.027473055 & 0.013871739 & 0.004259526 \\ 
		\bottomrule[1.5pt]
\end{tabular}
\end{table}

\begin{table}[H]   %[H]
\centering
\begin{tabular}{cccccc}
\toprule[1.5pt]
企业& 信誉评级 & $s_j^*$ & $s_j^0$ & $f_j^*$ & $f_j^*/(\sum_{j=1}^n f_j^*)$ \\ 
\midrule[1pt]
E80 & C & 1.95303243 & 0.027473055 & 0.013871739 & 0.004259526 \\ 
E81 & A & 1.95303243 & 0.027473055 & 0.013871739 & 0.004259526 \\ 
E84 & A & 1.95303243 & 0.027473056 & 0.013871739 & 0.004259526 \\ 
E85 & B & 1.95303243 & 0.027473078 & 0.01387175 & 0.004259529 \\ 
E86 & C & 1.95303243 & 0.027473055 & 0.013871739 & 0.004259526 \\ 
E87 & C & 1.95303243 & 0.027473055 & 0.013871739 & 0.004259526 \\ 
E88 & A & 1.95303243 & 0.027473058 & 0.013871741 & 0.004259526 \\ 
E90 & C & 1.95303243 & 0.027473058 & 0.013871741 & 0.004259526 \\ 
E91 & A & 1.95303243 & 0.027473055 & 0.013871739 & 0.004259526 \\ 
E92 & C & 1.95303243 & 0.027474603 & 0.01387251 & 0.004259762 \\ 
E93 & B & 1.95303243 & 0.027473068 & 0.013871745 & 0.004259528 \\ 
E94 & C & 1.95303243 & 0.030603949 & 0.015428205 & 0.004737462 \\ 
E95 & B & 1.95303243 & 0.882304039 & 0.311181424 & 0.095552929 \\ 
E97 & B & 1.95303243 & 0.027594962 & 0.013932435 & 0.004278163 \\ 
E98 & B & 1.95303243 & 0.027473101 & 0.013871762 & 0.004259533 \\ 
E104 & C & 1.95303243 & 1.905626954 & 0.493857261 & 0.151646288 \\ 
E105 & C & 1.95303243 & 0.049094737 & 0.024521288 & 0.00752963 \\ 
E106 & B & 1.95303243 & 0.027474071 & 0.013872245 & 0.004259681 \\ 
E110 & C & 1.95303243 & 0.558298705 & 0.222311864 & 0.068264196 \\ 
		\bottomrule[1.5pt]
\end{tabular}
\end{table}


\section{TOPSIS算法代码}
\begin{lstlisting}[language=matlab]
clear
A=load('A.txt');
B=A';%将导入的数据矩阵进行转置
[m,n]=size(B);
C=zscore(B);%数据标准化
C=C';
E=[1 1/3 1/4;3 1 7;4 1/7 1];%成对比较判断矩阵
[x,y]=eig(E);%x是特征向量矩阵,y是特征值矩阵
lamda=max(diag(y));%求最大特征值
[max_column, index_row] = max(y);%最大特征值所在位置
w=x(:,index_row(2));%对应特征向量
w=w';
w=repmat(w,n,1);
E=C.*w;%计算加权属性
cstar=max(E);%求正理想解
C0=min(E);%求负理想解
for i = 1:n
sstar(i)=norm(D(1,:)-cstar)%求到正理想解的距离
s0(i)=norm(D(i,:)-D0)%求到负理想解的距离
end
f=s0./(sstar+s0);
xlswrite('a1.xls',[sstar' s0' f'])%把计算结果写入excel中
 \end{lstlisting}

\section{计算成对比较判断矩阵的特征值和特征向量代码}

\begin{lstlisting}[language=matlab]
clear
A=load('A.txt');
B=A';%将导入的数据矩阵进行转置
C=zscore(B);%数据标准化
E=[1 1/3 1/4;3 1 7;4 1/7 1];%成对比较判断矩阵
[x,y]=eig(E);%x是特征向量矩阵,y是特征值矩阵
lamda=max(diag(y));%求最大特征值
[max_column, index_row] = max(y);%最大特征值所在位置
w=x(:,index_row(2));%对应特征向量
 \end{lstlisting}

\section{银行最大利益下的寻求最优客户流失率组合的代码}
\begin{lstlisting}[language=c++]
#include <stdio.h>
#include<iostream>

using namespace std;

double x[29]={0,0.094574126,0.135727183,0.224603354,0.302038102,0.347315668,
0.41347177,0.447890973,0.497634453,0.511096612,0.573393087,0.609492115,
0.652944774,0.667541843,0.694779921,0.708302023,0.731275401,0.775091405,
0.79822736,0.790527266,0.815196986,0.814421029,0.854811097,0.870317343,
0.871428085,0.885925945,0.874434682,0.902725909,0.922060687};
double y[29]={0,0.0667995830000000,0.135052060000000,0.206580080000000,
0.276812293000000,0.302883401000000,0.370215852000000,0.406296668000000,
0.458295295000000,0.508718692000000,0.544408837000000,0.548493958000000,
0.588765696000000,0.625764576000000,0.635605146000000,0.673527424000000,
0.696925431000000,0.705315993000000,0.742936326000000,0.776400729000000,
0.762022595000000,0.791503697000000,0.814998933000000,0.822297861000000,
0.835301602000000,0.845747745000000,0.842070844000000,0.868159536000000,
0.885864919000000};
double z[29]={0,0.0687253060000000,0.122099029000000,0.181252146000000,
0.263302863000000,0.290189098000000,0.349715590000000,0.390771683000000,
0.457238070000000,0.492660433000000,0.513660239000000,0.530248706000000,
0.587762408000000,0.590097045000000,0.642993656000000,0.658839416000000,
0.696870573000000,0.719103552000000,0.711101237000000,0.750627656000000,
0.776816043000000,0.784480512000000,0.795566274000000,0.820051434000000,
0.832288422000000,0.844089875000000,0.836974326000000,0.872558957000000,
0.895164739000000};
double q[29]={0.04,0.0425,0.0465,0.0505,0.0545,0.0585,0.0625,0.0665,0.0705,
0.0745,0.0785,0.0825,0.0865,0.0905,0.0945,0.0985,0.1025,0.1065,0.1105,
0.1145,0.1185,0.1225,0.1265,0.1305,0.1345,0.1385,0.1425,0.1465,0.15};
double x0;
double y1;
double z0;
double d;
double sumxyz;
double max_sum;
double a;
double b;
double c;
double m,n,t;
int main()
{
	d=1;
	a=0;
	b=0;
	c=0;
	max_sum=0;
	x0=304388777;
	y1=694272067;
	z0=1339156;
	for(int i=0;i<29;i++)
	{
		for(int j=0;j<29;j++)
		{
			for(int k=0;k<29;k++)
			{
				m=(d-x[i])*x0;
				n=(d-y[j])*y1;
				t=(d-z[k])*z0;
				sumxyz=m*q[i]+n*q[j]+t*q[k];
				if(sumxyz>max_sum)
				{
					max_sum=sumxyz;
					a=x[i];
					b=y[j];
					c=z[k];
				}
			}
		}	
	}
	cout<<"a="<<a<<" b="<<b<<" c="<<c<<endl;
	cout<<max_sum<<endl;
	return 0;
}
\end{lstlisting}

\section{模糊综合评价模型代码}
\begin{lstlisting}[language=matlab]
clear
R=load('GTJY.txt');
Q=[0.15 0.4 0.15 0.3];
B=Q*R;
xlswrite('gtjy.xls',B)
\end{lstlisting}

\section{支撑文件列表}
\begin{lstlisting}
(文件夹:问题二)
贷款配比.txt
第二问贷款额分配(过渡数据).xlsx
第二问贷款分配的迭代过程数据.xlsx
第二问贷款分配的最终结果.xlsx
银行最人利润代码.cpp

(文件夹:问题三)
企业初步分类.xlsx
WENTl3.m
各类型企业的模糊判断矩阵.xlsx
各企业调整分配额_结果.xlsx
信贷额分配调整过程表.xlsx
第三问数据处理表.xlsx
第三问企业分类表.xlsx

(文件夹:问题一的第二个模型)
平均年度需贷款额_信贷得分_收益比.xlsx

(文件夹:问题一的第一个模型)
A.txt
B.txt
C.txt
D.txt
tezhengjuzhen_tezhengxiangliang.m
TOPSIS.m
信誉评级为A的企业各项数据处理结果.xlsx
信誉评级为B的企业各项数据处理结果.xlsx
信誉评级为C的企业各项数据处理结果.xlsx
信誉评级为D的企业各项数据处理结果.xlsx
\end{lstlisting}

\end{appendices}


\end{document} 